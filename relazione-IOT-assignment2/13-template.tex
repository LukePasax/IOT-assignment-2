\documentclass[a4paper,12pt]{report}

\usepackage{alltt, fancyvrb, url}
\usepackage{graphicx}
\usepackage[utf8]{inputenc}
\usepackage{float}
\usepackage{hyperref}

% Questo commentalo se vuoi scrivere in inglese.
\usepackage[italian]{babel}

\usepackage[italian]{cleveref}

\title{relazione per\\``IOT-assignment-2''}

\author{Andrea Zavatta - Lorenzo Tosi - Luca Pasini}
\date{\today}


\begin{document}

\maketitle

\tableofcontents

\chapter{Analisi}

Abbiamo deciso in questo capitolo di descrivere il dominio applicativo e di rappresentare cosa l'applicazione dovrà fare attraverso la 'Macchina a stati finiti', in questo modo riusciamo a rappresentare il comportamento del sistema in questione.
%
\section{Macchina a stati finiti}

In particolare l'applicazione in questione farà uso di una macchina a stati finiti non deterministica, questo perchè uno stato presenta più di una possibile computazione per determinati input dai diversi componenti.
%
Come possiamo notare dalle specifiche possiamo contraddistinguere 3 stati principali.

\subsection*{Stato Normale}
\begin{itemize}
	\item Si fornisce una descrizione in linguaggio naturale di ciò che il software dovrà fare.
	\item Gli obiettivi sono spiegati con chiarezza, per punti.
	\item Se il software è stato commissionato o è destinato ad un utente o compagnia specifici, il committente viene nominato.
	\item Se vi sono termini il cui significato non è immediatamente intuibile, essi vengono spiegati.
	\item Vengono descritti separatamente requisiti funzionali e non funzionali.
	\item Considerato a un paio di pagine un limite ragionevole alla lunghezza della parte sui requisiti, in quello spazio si deve cercare di chiarire \textit{tutti} gli aspetti dell'applicazione, non lasciando decisioni che impattano la parte ``esterna'' alla discussione del design (che dovrebbe solo occuparsi della parte ``interna'').
\end{itemize}

\subsection*{Stato di preallarme}
\begin{itemize}
	\item Si forniscono indicazioni circa le soluzioni che si vogliono adottare
	\item Si forniscono dettagli di tipo tecnico o implementativo (parlando di classi, linguaggi di programmazione, librerie, eccetera)
\end{itemize}

\subsection*{Stato di Allarme}
Il software, commissionato dal gestore del centro di ricerca ``Aperture Laboratories Inc.''\footnote{\url{http://aperturescience.com/}}, mira alla costruzione di una intelligenza artificiale di nome GLaDOS (Genetic Lifeform and Disk Operating System).
%
Per intelligenza artificiale si intende un software in grado di assumere decisioni complesse in maniera semi autonoma sugli argomenti di sua competenza, a partire dai vincoli e dagli obiettivi datigli dall'utente.

\chapter{Design}

In questo capitolo si spiegano le strategie messe in campo per soddisfare i requisiti identificati nell'analisi.

Si parte da una visione architetturale, il cui scopo è informare il lettore di quale sia il funzionamento dell'applicativo realizzato ad alto livello.
%
In particolare, è necessario descrivere accuratamente in che modo i componenti principali del sistema si coordinano fra loro.
%
A seguire, si dettagliano alcune parti del design, quelle maggiormente rilevanti al fine di chiarificare la logica con cui sono stati affrontati i principali aspetti dell'applicazione.

\section{Architettura}

Questa sezione spiega come le componenti principali del software interagiscono fra loro.
%
In particolare, qui va spiegato \textbf{se} e \textbf{come} è stato utilizzato il pattern
architetturale model-view-controller (e/o alcune sue declinazioni specifiche, come entity-control-boundary).

Se non è stato utilizzato MVC, va spiegata in maniera molto accurata l'architettura scelta, giustificandola in modo appropriato.

Se è stato scelto MVC, vanno identificate con precisione le interfacce e classi che rappresentano i punti d'ingresso per modello, view, e controller.
Raccomandiamo di sfruttare la definizione del dominio fatta in fase di analisi per capire quale sia l'entry point del model, e di non realizzare un'unica macro-interfaccia che, spesso, finisce con l'essere il prodromo ad una ``God class''.
%
Consigliamo anche di separare bene controller e model, facendo attenzione a non includere nel secondo strategie d'uso che appartengono al primo.

In questa sezione vanno descritte, per ciascun componente architetturale che ruoli ricopre (due o tre ruoli al massimo), ed in che modo interagisce (ossia, scambia informazioni) con gli altri componenti dell'architettura.
%
Raccomandiamo di porre particolare attenzione al design dell'interazione fra view e controller: se ben progettato, sostituire in blocco la view non dovrebbe causare alcuna modifica nel controller (tantomeno nel model).

Con questa architettura, possono essere aggiunti un numero arbitrario di input ed output
all'intelligenza artificiale.
%
Ovviamente, mentre l'aggiunta di output è semplice e non richiede alcuna modifica all'IA, la
presenza di nuovi tipi di evento richiede invece in potenza aggiunte o rifiniture a GLaDOS.
%
Questo è dovuto al fatto che nuovi Input rappresentano di fatto nuovi elementi della business
logic, la cui alterazione od espansione inevitabilmente impatta il controller del progetto.

\section{Design Pattern}

Questa sezione spiega come le componenti principali del software interagiscono fra loro.
%
In particolare, qui va spiegato \textbf{se} e \textbf{come} è stato utilizzato il pattern
architetturale model-view-controller (e/o alcune sue declinazioni specifiche, come entity-control-boundary).

Se non è stato utilizzato MVC, va spiegata in maniera molto accurata l'architettura scelta, giustificandola in modo appropriato.

Se è stato scelto MVC, vanno identificate con precisione le interfacce e classi che rappresentano i punti d'ingresso per modello, view, e controller.
Raccomandiamo di sfruttare la definizione del dominio fatta in fase di analisi per capire quale sia l'entry point del model, e di non realizzare un'unica macro-interfaccia che, spesso, finisce con l'essere il prodromo ad una ``God class''.
%
Consigliamo anche di separare bene controller e model, facendo attenzione a non includere nel secondo strategie d'uso che appartengono al primo.

In questa sezione vanno descritte, per ciascun componente architetturale che ruoli ricopre (due o tre ruoli al massimo), ed in che modo interagisce (ossia, scambia informazioni) con gli altri componenti dell'architettura.
%
Raccomandiamo di porre particolare attenzione al design dell'interazione fra view e controller: se ben progettato, sostituire in blocco la view non dovrebbe causare alcuna modifica nel controller (tantomeno nel model).

Con questa architettura, possono essere aggiunti un numero arbitrario di input ed output
all'intelligenza artificiale.
%
Ovviamente, mentre l'aggiunta di output è semplice e non richiede alcuna modifica all'IA, la
presenza di nuovi tipi di evento richiede invece in potenza aggiunte o rifiniture a GLaDOS.

\section{Grafico}

Questa sezione spiega come le componenti principali del software interagiscono fra loro.
%
In particolare, qui va spiegato \textbf{se} e \textbf{come} è stato utilizzato il pattern
architetturale model-view-controller (e/o alcune sue declinazioni specifiche, come entity-control-boundary).

Se non è stato utilizzato MVC, va spiegata in maniera molto accurata l'architettura scelta, giustificandola in modo appropriato.

Se è stato scelto MVC, vanno identificate con precisione le interfacce e classi che rappresentano i punti d'ingresso per modello, view, e controller.
Raccomandiamo di sfruttare la definizione del dominio fatta in fase di analisi per capire quale sia l'entry point del model, e di non realizzare un'unica macro-interfaccia che, spesso, finisce con l'essere il prodromo ad una ``God class''.
%
Consigliamo anche di separare bene controller e model, facendo attenzione a non includere nel secondo strategie d'uso che appartengono al primo.

In questa sezione vanno descritte, per ciascun componente architetturale che ruoli ricopre (due o tre ruoli al massimo), ed in che modo interagisce (ossia, scambia informazioni) con gli altri componenti dell'architettura.
%
Raccomandiamo di porre particolare attenzione al design dell'interazione fra view e controller: se ben progettato, sostituire in blocco la view non dovrebbe causare alcuna modifica nel controller (tantomeno nel model).

Con questa architettura, possono essere aggiunti un numero arbitrario di input ed output
all'intelligenza artificiale.
%
Ovviamente, mentre l'aggiunta di output è semplice e non richiede alcuna modifica all'IA, la
presenza di nuovi tipi di evento richiede invece in potenza aggiunte o rifiniture a GLaDOS.


\end{document}
